% Abstände zwischen Numerierung und Einträgen
% im Tabellen- und Abbildungsverzeichnis.
\DeclareTOCStyleEntry[numwidth=1.35cm]{tocline}{subsection}
\DeclareTOCStyleEntry[numwidth=1.35cm, raggedentrytext=true]{tocline}{table}
\DeclareTOCStyleEntry[numwidth=1.35cm, raggedentrytext=true]{tocline}{figure}

\renewcommand{\hrefsection}[2]{\clearpage\hypertarget{#1}{\section{#2}\label{#1}}}

% Makro für Verweise auf andere Teile des Dokuments, wobei der Text
% des Links aus dem Kürzel des übergebenen Labels expandiert wird.
% vgl. ursprüngluiche Definition in common/common-macros.tex
\renewcommand{\tdslink}[2][no]{\hyperlink{#2}{\secitemformat{\directlua{replacelabel('#2', '#1')}}}}

\renewcommand{\sfrlink}[2][]{%
  \index{\sfrplain{\directlua{removeSfrSubComponent('#2')}}@\sfr{\directlua{removeSfrSubComponent('#2')}}}%
  \hyperlink{\directlua{removeSfrSubComponent('#2')}}{\sfr[#1]{#2}}}

\newcommand{\sfrlinknoindex}[2][]{%
  \hyperlink{\directlua{removeSfrSubComponent('#2')}}{\sfr[#1]{#2}}}

\newcommand{\getModuleStatus}[1]{\directlua{get_module_status('#1')}}

\newcommand{\printModuleToSfrTable}[2][enf]{\directlua{print_module_to_sfr_table('#2', '#1')}}

\newcommand{\printModulesForSfrRows}[1]{\directlua{print_modules_for_sfr_rows('#1')}}

\newenvironment{enfsfrtable}%
{\begin{tabular*}{0.9\textwidth}{@{}lll@{}}\toprule}
   {\bottomrule\end{tabular*}}

\newcommand{\printSubsystemToSfrTable}[2][enf]{\directlua{print_subsys_to_sfr_table('#2', '#1')}}

\newenvironment{enfsfrsubsystable}%
{\begin{tabular*}{\textwidth}{@{}p{0.3\textwidth}L{0.68\textwidth-\tabcolsep}@{}}\toprule}
   {\bottomrule\end{tabular*}}

\newcommand{\printModuleToBundleTable}[1]{\directlua{print_module_to_bundle_table('#1')}}

\newenvironment{bundletable}%
{\begin{tabular*}{0.9\textwidth}{@{}l@{}}\toprule}
   {\bottomrule\end{tabular*}}


%%% Local Variables:
%%% mode: latex
%%% TeX-master: "adv_tds"
%%% TeX-engine: luatex
%%% End:
