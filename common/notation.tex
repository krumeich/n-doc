\section*{Remarks on Notation}

Table~\vref{tab:intro.notc} lists the typographical conventions used in this
document and their usage. Often the categories are not as clear-cut as they
seem. Occasionally, the same term is used on different levels of abstraction. It
is not always possible to determine that level at first glance. For example, a
term can be both a keyword taken from the specification and the name of a
variable in the code. Typographic consistency supports the reader in determining
the level of abstraction. The explanations in \autoref{tab:intro.notc} shall
motivate the differentiation.

Subsystems and modules are separated by a double colon, as in
\secitemformat{Sub\-sys\-tem::Mo\-dule}. Interfaces of modules are separated by
two forward slashes: \secitemformat{Sub\-sys\-tem::Mo\-dule//Interface}.

The quoted names of code elements -- especially of the Java-based components --
tend to be quite long. In such cases, hyphens are used to split the name across
lines, . This avoids extraneous amounts of white space and supports readability.

\begin{table}[htb]
  \centering{}
  \begin{tabularx}{\textwidth}{@{}lX@{}}
    \toprule
    Typographic Markup & Purpose \\
    \midrule
    \keyword{keywords} & \keyword{Keywords} are terms that are directly taken from the specification. This can be the names of configuration parametes and their values. Also other highly specific terms can be typeset as \keyword{keywords}.\\
    \code{code elements} & \code{code elements} are terms that are taken from the TOE's source code. This can be names of classes, methods and other types, but also their parameters or logical structures of the programming language employed by the TOE.\\
    \filename{file names} & \filename{file names} relate to names of file system elements, such as files or directories.\\
    \secitem{security terms} & Terms that are directly related to the Common Criteria framework are typeset as  \secitem{security terms}. This can be SFRs or the names of SF and TSFI. Furthermore, the names of subsystems and modules that constitute the TOE are typeset in this way. This holds also for the names of certficiates and other key material.\\
    \bottomrule
  \end{tabularx}
    \caption{Typographic Conventions}
    \label{tab:intro.notc}
\end{table}


%%% Local Variables:
%%% mode: latex
%%% TeX-master: shared
%%% TeX-engine: luatex
%%% End:
