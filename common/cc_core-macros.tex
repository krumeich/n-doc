\makeatletter{}

% Macro for "Enforcing"
\newcommand{\enfc}{\secitemfont{SFR-enforcing}}
% Macro for "Supporting"
\newcommand{\supp}{\secitemfont{SFR-supporting}}
% Macro for "non-Interfering"
\newcommand{\noninterfering}{\secitemfont{non-interfering}}
% Macro for "non-TSF"
\newcommand{\nontsf}{\secitemfont{non-TSF}}

% Macros for resolving names of security objectives. The parameter references
% the database entries defined in common/db/obj.csv
\newcommand{\obj}[1]{\secitemformat{\directlua{getObjective('#1')}}}
\newcommand{\objtext}[1]{\directlua{getObjectiveText('#1')}}
\newcommand{\objplain}[1]{\directlua{getObjective('#1')}}
\newcommand{\objlink}[1]{\hyperlink{\directlua{toLower('#1')}}{\obj{#1}}}

% Macros for resolving names of security problems. The parameter references
% the database entries defined in common/db/spd.csv
\newcommand{\spd}[1]{\secitemformat{\directlua{getSpd('#1')}}}
\newcommand{\spdtext}[1]{\directlua{getSpdText('#1')}}
\newcommand{\spdsource}[1]{\directlua{getSpdSource('#1')}}
\newcommand{\spdlink}[1]{\hyperlink{\directlua{toLower('#1')}}{\spd{#1}}}

% Macros for resolving names of subjects and objects (in adv_tds). The parameter
% references the database entries defined in common/db/subjobj.csv
\newcommand{\subjobj}[1]{\secitemformat{\directlua{getSubjobj('#1')}}}
\newcommand{\subjobjtext}[1]{\directlua{getSubjobjText('#1')}}
\newcommand{\subjobjlink}[1]{\hyperlink{\directlua{toLower('#1')}}{\subjobj{#1}}}


%% ------------------------------------------------------------
%%
%% Macros for
%% Security Functional Requirements (SFR)
%%

% Macros for resolving names of security objectives. The required parameter
% references the database entries defined in common/db/sfr.csv. The optional
% parameter is used to specify sub-elements: FDP_ACF.1.2(2)

% Prints the SFR with no formatting
\newcommand{\sfrplain}[2][]{\directlua{getSfr('#2')}#1}

% Formats an SFR. Macro is seldom used by the user.
\newcommand{\formatsfr}[1]{{\secitemformat{#1}}}

% Prints the SFR and formats the SFR.
\newcommand{\sfr}[2][]{\formatsfr{\sfrplain[#1]{#2}}}

% Prints the SFR and formats the SFR. Also creates a Hyperlink
\newcommand{\sfrlink}[2][]{\hyperlink{\sfrmain{\directlua{toLower('#2')}}}{\sfr[#1]{#2}}}

% Prints the SFR without its sub-components.
% FDP_ACF.1.2 -> FDP_ACF.1
\newcommand{\sfrmain}[1]{\directlua{removeSfrSubComponent('#1')}}

% Prints the text associated with the SFR
\newcommand{\sfrtext}[1]{\directlua{getSfrText('#1')}}

% Used for printing indices. Not intended for the end user
\newcommand{\sfrindex}[1]{\textsmaller[1.5]{\secitemfont{\sfrplain{\sfrmain{#1}}}}}

% Macro for implemented/supported SFR. Paramter is in form of table lines, with
% three items max. Can be used during process descriptions in adv_tds or in
% ase_tss
\newcommand{\implementedsfr}[1]{\enfsupsfrtable{\ndoc@implementedsfr}{#1}}
\newcommand{\supportedsfr}[1]{\enfsupsfrtable{\ndoc@supportedsfr}{#1}}

% Not to be used by the user
\newcommand{\enfsupsfrtable}[2]{
  \begin{flushright}
    \minibox[frame,l,t]{{\small \textsf{#1}}\\
      \begin{tabular}{@{}lll@{}}
        #2
      \end{tabular}}
  \end{flushright}}

%% ------------------------------------------------------------
%%
%% Macros for
%% TOE Security Function Interfaces (TSFI)
%%

% Macros for resolving names of TSFI. The parameter
% references the database entries defined in common/db/tsfi.csv
\newcommand{\formatintf}[1]{{\ccfont #1}}
\newcommand{\intf}[1]{\secitemformat{#1}}
\newcommand{\tsfi}[1]{\intf{\directlua{getTsfi('#1')}}}
\newcommand{\tsfilink}[1]{\hyperlink{tsfi.#1}{\tsfi{#1}}}

%% ------------------------------------------------------------
%%
%% Macros for
%% TOE Security Functions (SF)
%%

% Macros for resolving names of TOE Security Functions (SF). The parameter
% references the database entries defined in common/db/sf.csv
\newcommand{\secfunc}[1]{\formatsecfunc{\directlua{getSecfunc('#1')}}}
\newcommand{\secfuncplain}[1]{\directlua{getSecfunc('#1')}}
\newcommand{\secfunctext}[1]{\directlua{getSecfuncText('#1')}}
\newcommand{\secfunclink}[1]{\hyperlink{\directlua{toLower('#1')}}{\secfunc{#1}}}
\newcommand{\formatsecfunc}[1]{{\secitemformat{#1}}}

% Creates a headline based on a key in sf.csv
\newcommand{\secfuncheadline}[1]{\texorpdfstring%
  {\secfunctext{#1}\hspace{1em}(\secitemfont{\secfuncplain{#1}})}%
  {\secfunctext{#1} (\secfuncplain{#1})}}

% Macros for referencing TDS subsystems, modules and interfaces.  The obligatory
% parameter denotes the key defined in
% common/db/{subsystems,modules,interfaces}.csv, e.g.
% "sub.vpn", "mod.vpn.core" or "int.vpn.core.connect"
\newcommand{\tds}[2][no]{\secitemformat{\directlua{replacelabel('#2', '#1')}}}
\newcommand{\tdsplain}[2][no]{\directlua{replacelabel('#2', '#1')}}
\newcommand{\tdsplaintext}[2][no]{\directlua{replacelabelplain('#2', '#1')}}

% Creates a link to a TDS entity (see above). Usually only relevant within TDS.
% Defined here for compatibility reasons. Macro is redefined in
% adv_tds/user_macros to produce a hyperlink
\newcommand{\tdslink}[2][no]{\tds[#1]{#2}}


\makeatother{}


%%% Local Variables:
%%% mode: latex
%%% TeX-master: shared
%%% TeX-engine: luatex
%%% End:
